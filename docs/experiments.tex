\section{Experiments}

The experiments section shows empirical evaluations of the proposed solution.
Again, it's not enough to just list the results.
You need to discuss them and show their implications.
For example, 
\begin{itemize}
    \item motivations behind experiments (why you run such experiment?)
    \item reasons behind the success and failure of solutions.
    \item limitations of the current experiment configuration and 
        possible improvements in future work.
\end{itemize}

Typically, the experiment section contains the following parts,
\begin{itemize}
    \item a subsection (paragraphs) on describing the evaluation environment,
        which includes 
    \begin{itemize}
        \item datasets (statistics, train/dev/test split, etc).
        \item criteria (e.g., when judge a correct entity in an entity detection task, 
            specifying ``exactly match'' or ``overlapped match'', 
            whether you consider entity types, etc.).
        \item default settings of your models.
        \item hyper-parameters (may go to the supplementary)
    \end{itemize}
    \item a subsection on baselines, including 
        short summaries of their models (echo the introduction/related work section)
        and the reasons you choose them for comparison.
    \item a subsection on results and discussions (the main part).
    \item a subsection on error analysis (for empirical methods) 
        which gives concrete examples to illustrate pros and cons of your model.
\end{itemize}




 


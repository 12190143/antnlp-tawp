\section{Introduction}

The introduction section is the most important part in your writing.
You may think that it is the compass which guides the tired readers to 
see the light of your paper in the field.
And you may also put yourself in the position of a tough reviewer,
and ask how you would judge the paper after going through this section.

In general, the introduction tells a story about your research problem: 
how it comes out, why it is worth studying, 
where is its position in a bigger picture,
how you figure out your solutions etc.
Typically, it contains
\begin{itemize}
    \item one paragraph on the problem's definition and its background. 
     Why it is an interest and important problem to investigate?
    \item one or two paragraphs on challenges. 
     What makes your problem hard?
     Why naive solutions fail? 
     What are ignored aspects in previous solutions?
    \item one figure/table/example to illustrate the challenges, 
    the failure of existing systems,
    the differences to previous methods etc.
    \item one or two paragraphs on your solution. 
     How your method different with previous ones?
     Why it is able to better tackle the challenges?
    \item one paragraph on experimental results. 
     On which datasets you test your system?
     How about the results? 
     Do they demonstrate the advantages of your solution?
    \item one paragraph summarizing contributions in bullet form.
\end{itemize}


Writing a good introduction is hard. 
According to our practice, some tips may help you accomplishing it
(also applied to other sections),
\begin{itemize}
    \item the coarse-to-fine strategy: 
    for each paragraph, writing a piece of note on what should be mentioned in it. 
    After doing this for all paragraphs 
    (now you should have a big picture of your story), 
    you can fulfill each paragraph by thinking about how to write 
    the first/second/.../last sentence to cover contents in the paragraph note,
    and then you could start to organize words.
    \item the first and the last sentence of a paragraph are important.
    Usually, they should reflect the main topic of the paragraph and connections
    with previous/next paragraphs.
    \item to be well-organized, you may consider inserting conjunctions:
    they could make the underlying logic of texts more explicit.
    (e.g., however, but, although, even, besides, despite, furthermore,
    on the one/other hand ...)
\end{itemize}



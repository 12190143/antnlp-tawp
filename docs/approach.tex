\section{Approach}

The approach section contains models, algorithms, system descriptions, and theorems.
They are almost technical materials.
However, it doesn't mean you only need to describe the formulations and models.
Illustrating the rational behind them is more important.
Why you design your model this way? 
What are alternatives and why you don't choose them?
What are advantages and drawbacks of your solution?
At the same time, 
it is important to echo the statements in the introduction section
(i.e., challenges of the problem, advantages of your approach etc).
Remember that
\textbf{every section of your paper should tell a story}.

The approach section could contains
\begin{itemize}
    \item a subsection on problem/task definition,
    \item an (optional) subsection on preliminary models/algorithms,
    \item subsections on your models.
    \end{itemize}
Usually, you need two key supporting components in the approach section, 
symbols and figures.

\subsection{Symbols}
Please remember that 
symbols are used to make the paper easier, rather than harder, to digest.
Thus, the first principle on the symbol system is \textbf{``the simpler the better''}.
For example,
\begin{itemize}
    \item \emph{when it's possible to use natural language, 
        you may prefer not to use symbols. }
        For example, 
        ``if a word is not in the vocabulary'' is better than
        ``if $w \notin V$''.
    \item \emph{when it's possible to reuse an existing symbol, don't define a new symbol.}
        For example, if a sentence is denoted by $s$, 
        you may use $|s|$ to denote its length rather than a new symbol $n$.
        If the ground truth class label is $y$, you would prefer $y'$, 
        rather than a new notation $z$, to represent other class labels.
\end{itemize}

A good symbol system has three properties,
\begin{itemize}
    \item \textbf{Clear}. The meaning of every notation should be defined, 
        and there should be no conflict/ambiguous usage. Common mistakes include 
        \begin{itemize}
            \item using one notation to represent different things 
                (e.g., $w$ refers to both a word and a weight vector),
            \item using notations without definition,
            \item defining a notation but never use it,
            \item using inconsistent typesetting (e.g., $w$ or $\mathbf{w}$).
        \end{itemize}
    %In order to make a clear symbol system, you can maintain a table during your writing.
    \item \textbf{Informative}. Like good variable names,
        informative notations could make your reader's life easier. 
        For example, the leading character is a good notation for naming an object 
        (a \underline{s}entence $s$, a \underline{w}ord $w$, an \underline{ent}ity $ent$).
        You may also consider relations among different notations
        (e.g., using coherent symbols for related objects (``a vector $v$ in a set $V$'')).
    \item \textbf{Conventional}. Commonly, upper case letters represent sets, matrices, 
        and lower case letters represents scalars, indices. Following the convention
        may reduce the risk of misunderstanding.
        Meanings of some notations may also have strong prior.
        For example, ``$d, \delta, \Delta$'' usually imply ``differences'', and
        ``$z$'' usually stands for unobserved variables in inference models.
\end{itemize}

\subsection{Figures}
Please make figures accurate, consistent with texts, and beautiful.


%footnotes and paratheses



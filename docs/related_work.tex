\section{Related Work}

The related work section, which usually has an underestimated importance,
is not simply a bunch of citations and loosely organized descriptions.
A well-written related work gives plenty of insights 
on what have happened in the field.
For experts, it should reveal key technical differences 
between your solution and prior efforts.
For newcomers,
it should tell a brief story on the background 
and connections with other tasks.

One common practice is to organize related work into different groups.  
You may consider
\begin{itemize}
    \item different definitions of the problem 
    (e.g., for reading comprehension tasks, there are
     cloze test, multiple choice and directly extracting answer spans).
    \item different models
    (e.g., pipeline/joint model, fully/weakly supervised,
     local/global model, domain specific/independent...).
    \item different key components 
    (e.g., w/o external knowledge, w/o specific preprocessing...).
\end{itemize}
Instead of giving a thorough survey, 
you shall focus on those work closely related to your paper. 
For example, if your approach is supervised,
you may emphasize related supervised models,
and briefly mention (or just ignore) unsupervised/semi-supervised methods.
Again, a coarse-to-fine strategy could be applied:
you can first decide the groups, then list important papers in each group,
and then describe individual papers.


A final remark on writing related work is the usage of \texttt{citet} and 
\texttt{citep}
\footnote{In package natbib}.
We suggest that instead of 
\begin{quote}
\citep{miwa-sasaki:2014:EMNLP2014} proposes a joint model...
\end{quote}
you use
\begin{quote}
\citet{miwa-sasaki:2014:EMNLP2014} propose a joint model...
\end{quote}
One correct usage of \texttt{citep} is 
\begin{quote}
We compare our model with the joint model proposed in 
\citep{miwa-sasaki:2014:EMNLP2014}.
\end{quote}




